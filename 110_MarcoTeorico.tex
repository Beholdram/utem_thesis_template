%%%%%%%%%%%%%%%%%%%%%%%%%%%%%%%%%%%%%%%%%%%%%%%%%%%%%%%%%%%%%%%%%%%%%%%%%%%%%%%
%                                                                             %
%                                                                             %
%                                                                             %
% Cristian Recabarren 2016                                                    %
%                                                                             %
%%%%%%%%%%%%%%%%%%%%%%%%%%%%%%%%%%%%%%%%%%%%%%%%%%%%%%%%%%%%%%%%%%%%%%%%%%%%%%%

%-------------------------------------------------------------------------
\chapter{Marco Teórico y Metodología}
\label{chap:MarcoTeorico}
%-------------------------------------------------------------------------
%\begin{itemize}
% \item Definir el rango de longitud de onda a trabajar.
% \item Estudiar la estructura del receptor de radio.
% \item Estudiar la estructura de las señales capturadas por el receptor de radio.
% \item Estudiar técnicas existentes para digitalización de señales.
% \item Estudiar las técnicas y estructuras existentes para el almacenamiento de datos científicos.
% \item Identificar las diversas aplicaciones disponibles para reducción de datos astronómicos.
% \item Implementar una herramienta de reducción de datos según la estructura estudiada. 
%\end{itemize}

\section{Introducción}
\label{sec:introduction}


