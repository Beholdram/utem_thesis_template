%%%%%%%%%%%%%%%%%%%%%%%%%%%%%%%%%%%%%%%%%%%%%%%%%%%%%%%%%%%%%%%%%%%%%%%%%%%%%%%
%                                                                             %
%                                                                             %
%                                                                             %
% Cristian Recabarren 2016                                                    %
%                                                                             %
%%%%%%%%%%%%%%%%%%%%%%%%%%%%%%%%%%%%%%%%%%%%%%%%%%%%%%%%%%%%%%%%%%%%%%%%%%%%%%%

%-------------------------------------------------------------------------
\chapter{Introducción}
\label{chap:introduction}

\section{Descripción de la problemática}


\section{Motivación}
\label{sec:motivation}


\section{Objetivos}
\label{sec:objectives}


\section{Alcances}
\label{sec:scope}


\section{Metodología}
\label{sec:methodology}


\section{Estructura}
\label{sec:structure}
El presente trabajo se estructurará en cinco capítulos, los cuales engloban las grandes macro tareas que corresponde al proyecto a desarrollar.

El capítulo~\ref{chap:MarcoTeorico} comprende el marco teórico, donde se describe la teoría y estado del arte, necesarios para sentar las bases sobre las cuales se sustenta la etapa de~\nameref{chap:research}. El trabajo sintetiza y resume contenido de distintas fuentes bibliográficas.

El capítulo~\ref{chap:research} comprende el desarrollo de la investigación, sistemas implementados; y en particular se detalla teóricamente el montaje utilizado para registrar datos y la relevancia que tienen estos para el análisis de la problemática.

En el capítulo~\ref{chap:resultsAnalysis} se presentan los resultados obtenidos y su correspondiente análisis, latencias y diferencias de rendimiento entre los distintos algoritmos encontrados, como también entre distintas configuraciones de hardware.

Finalmente, las conclusiones se presentan en el capítulo~\ref{chap:Conclusiones}, donde se analiza el trabajo desde un punto de vista más general, contrastando objetivos y resultados obtenidos, y a la vez proponiendo trabajos futuros.

Se incluyen adicionalmente anexos con la terminología y acrónimos, unidades y prefijos del Sistema Internacional de Medidas, entre otros datos de interés utilizados en el presente trabajo, cuyo objetivo es servir como una consulta rápida para facilitar la comprensión del mismo.

